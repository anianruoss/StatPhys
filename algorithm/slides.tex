\usepackage[english]{babel}
\usepackage[utf8]{inputenc}
\usepackage[T1]{fontenc}

\usepackage[scaled]{helvet} % font family helvetica, important for good looking slides
\usepackage{graphicx}
\renewcommand*\familydefault{\sfdefault}
\usepackage{hyperref}
\usepackage{adjustbox}
\usepackage{fancyvrb}

\usepackage{bbm}
\usepackage{tikz}
\usetikzlibrary{calc,shapes,shadows,arrows,patterns,decorations.pathreplacing,matrix,fit,shapes.geometric,shapes.multipart,fit,chains,scopes,overlay-beamer-styles}
\usepackage{pgfplots}
\usepackage{listings}
\usepackage{comment} % for being able to uncomment sections of the slides for different lectures
%\usepackage{filecontents}

\usepackage{etoolbox}
\usepackage[vlined]{algorithm2e}

\usepackage{booktabs}
\usepackage{colortbl}
% ---------------------------------------------------------
% beamer setup: minimalistic and clean
% ---------------------------------------------------------
\usetheme[height=1.2cm]{Rochester}
\usefonttheme{professionalfonts}
\setbeamertemplate{blocks}[rounded][shadow=true]
\setbeamertemplate{footline}{\hfill\insertframenumber\vskip1em} % without vskip it does not work -- the footline is too low for whatever reasons

\makeatother
\makeatletter\let\frametextheight\beamer@frametextheight\makeatother

% ----- additional vertical spaces in itemize -----
\setbeamertemplate{itemize/enumerate subbody begin}{\vspace{1ex}}
\setbeamertemplate{itemize/enumerate subbody end}{\vspace{1ex}}

% itemize with a bit more spaace between the items
\newenvironment{itemizew}{\itemize\addtolength{\itemsep}{0.5ex}}{\enditemize}
% itemize with left aligned itemize items
\newenvironment{bullets}{%
\settowidth{\leftmargini}{\usebeamertemplate{itemize item}}
\addtolength{\leftmargini}{\labelsep}
\addtolength{\itemsep}{0.5ex}
\itemize[]}
{\enditemize}


% ----- special fonts for headlines etc. -----
\newcommand{\foffont}{\usefont{OT1}{cmss}{bx}{n}} % alternative: bx (bold extended) -> sbc (semi bold condensed)
\newcommand{\algfont}{\usefont{OT1}{cmss}{x}{n}} % alternative: bx (bold extended) -> sbc (semi bold condensed)
\newcommand{\algcfont}{\usefont{OT1}{cmss}{x}{n}} % alternative: bx (bold extended) -> sbc (semi bold condensed)
\newcommand{\foffontx}{\usefont{OT1}{phv}{bc}{n}}
% normal font for code
\newcommand{\srcfont}{\usefont{T1}{lmtt}{b}{n}}
% narrow font for code
\newcommand{\srcfontn}{\usefont{T1}{lmvtt}{b}{n}}
% font for verbatim envirnoment. Note: must be fixed width!
\fvset{fontfamily=lmtt, fontseries=b,formatcom=\color{fcb}}

% ----- setting default fonts and colors -----
\usecolortheme{dolphin}
\definecolor{lightred}{rgb}{1,0.5,0.5}
\setbeamerfont*{frametitle}{size=\Large,series=\foffontx}
\setbeamerfont*{title}{series=\foffont, size=\LARGE}
\setbeamerfont*{author}{series=\foffont, size=\small}
\setbeamerfont*{institute}{series=\foffont, size=\normalsize}
\setbeamerfont*{date}{series=\foffont, size=\small}
\setbeamercolor*{institute}{fg=fcb}
\setbeamercolor*{author}{fg=fcb}
\setbeamercolor*{date}{fg=fcb}

\only<beamer>{\setbeamercolor{normal text}{fg=black, bg=white!95!black}\setbeamercolor*{title}{fg=black, bg=white!95!black}}
%\setbeamercolor{normal text}{fg=black, bg=white}
\setbeamertemplate{navigation symbols}{}
\setbeamertemplate{frametitle}{\insertframetitle}
\definecolor{lightblue}{RGB}{200,200,250}
\setbeamercolor*{frametitle}{fg=black, bg=lightblue} % set bg to green (or so) when adjusting..
\setbeamersize{text margin left = 0cm}
\setbeamercolor*{subtitle}{fg=gray}
\setbeamerfont*{subtitle}{series=\normalfont}
\setbeamercolor*{footline}{fg=gray}
\setbeamercolor{footnotemark}{fg=gray}
\setbeamercolor{footnote}{fg=gray}
\definecolor{lightgreen}{RGB}{173,255,47}
\definecolor{lightorange}{RGB}{255,165,0}
\definecolor{flatboxcolor}{named}{lightblue}
\definecolor{boxcolor}{RGB}{200,200,200}
\alt<beamer>{\definecolor{textbg}{RGB}{242,242,242}}{\definecolor{textbg}{RGB}{255,255,255}}
\definecolor{textfg}{named}{black}
\definecolor{textdimmed}{named}{lightgray}
\definecolor{texthighlighted}{named}{red}
\definecolor{texthighlightedx}{named}{blue}
\definecolor{fig}{named}{blue}

\newif\ifdark\darkfalse
\only<0|handout:0|trans:0>{\darktrue}% dark background for slides, white background for transparencies and handouts.

\ifcsname dark\endcsname \darktrue \fi
\ifcsname light\endcsname \darkfalse \fi
\ifdark
\setbeamercolor{title}{bg=black, fg=lightblue}
\setbeamercolor{frametitle}{bg=black!90!white, fg=lightblue}
\setbeamercolor{normal text}{fg=white, bg=black}
\setbeamercolor{background canvas}{bg=black}
\setbeamercolor{block body alerted}{bg=normal text.bg!90!black}
\setbeamercolor{block body}{bg=normal text.bg!90!normal text.fg}
\setbeamercolor{block body example}{bg=normal text.bg!90!black}
\setbeamercolor{block title alerted}{use={normal text,alerted text},fg=alerted text.fg!75!normal text.fg,bg=normal text.bg!75!black}
\setbeamercolor{block title}{bg=blue!50!normal text.bg}
\setbeamercolor{block title example}{use={normal text,example text},fg=example text.fg!75!normal text.fg,bg=normal text.bg!75!black}
\definecolor{flatboxcolor}{RGB}{0,0,50}
\definecolor{boxcolor}{RGB}{50,50,50}
\definecolor{textbg}{named}{black}
\definecolor{textfg}{named}{white}
\definecolor{textdimmed}{named}{darkgray}
\definecolor{dimmed}{named}{darkgray}
\definecolor{texthighlighted}{named}{red}
\definecolor{texthighlightedx}{named}{yellow}
\definecolor{fig}{RGB}{255,255,50}
\fi



\definecolor{bca}{RGB}{250,200,200}
\newcommand{\bca}{\color{bca}}
\definecolor{fca}{RGB}{180,80,80}
\newcommand{\fca}{\color{fca}}
\definecolor{bcb}{RGB}{200,200,250}
\newcommand{\bcb}{\color{bcb}}
\definecolor{fcb}{RGB}{80,80,180}
\newcommand{\fcb}{\color{fcb}}
\definecolor{bcc}{RGB}{250,250,200}
\newcommand{\bcc}{\color{bcc}}
\definecolor{fcc}{RGB}{150,150,80}
\newcommand{\fcc}{\color{fcc}}
\definecolor{bcd}{RGB}{250,200,250}
\newcommand{\bcd}{\color{bcd}}
\definecolor{fcd}{RGB}{150,100,150}
\newcommand{\fcd}{\color{fcd}}
\definecolor{bce}{RGB}{200,250,200}
\newcommand{\bce}{\color{bce}}
\definecolor{fce}{RGB}{80,150,80}
\newcommand{\fce}{\color{fce}}

\definecolor{texthighlighted}{RGB}{170,40,40}
\definecolor{texthighlightedx}{RGB}{40,40,170}
\definecolor{fig}{named}{fca}
\definecolor{hl}{named}{texthighlighted}
\definecolor{hlx}{named}{texthighlightedx}
\definecolor{hc}{named}{gray}
\definecolor{hlc}{named}{hc}

% shortcuts for text highlighting
\newcommand{\hl}{\color{hl}}
\newcommand{\hlx}{\color{hlx}}
\newcommand{\hc}{\color{hc}} % comment color
\newcommand{\dimm}{\color{gray}} % comment color
\newcommand{\thl}[1]{\textcolor{texthighlighted}{#1}}
\newcommand{\thlx}[2]{\textcolor{texthighlightedx}{#2}}
\newcommand{\emx}{\em\color{texthighlighted}} % emphasize even more
\newcommand{\hls}[1]{\text{\fcb\srcfont #1}}

% some geometric changes
\geometry{ paperwidth=\the\paperwidth, paperheight=\the\paperheight,
%hmargin=0.8ex,vmargin=0cm,head=0.2cm, foot=2.5cm
rmargin = 2ex,
lmargin = 2ex,
headsep=0.5ex
}
% adjust parskips in blocks
%\addtobeamertemplate{block begin}{}{\setlength{\parskip}{0cm}}
\setlength{\parskip}{0.5ex}

%adjust vertical distance of text to header on each slide
\setbeamertemplate{frametitle}
{
    \nointerlineskip
    %\begin{beamercolorbox}[sep=0cm,ht=0em,wd=\paperwidth]{frametitle}
       % \vbox{}\vskip-2ex%
        \vspace{-1.6em} \strut\insertframetitle%\strut
         %\vskip-0.8ex%
    %\end{beamercolorbox}
    \vspace{-0cm}
}

%----- adjust footnote location ---
%\addtobeamertemplate{footnote}{}{\vspace{8em}}
\let\oldfootnotesize\footnotesize
\renewcommand*{\footnotesize}{\oldfootnotesize\scriptsize}

% ------ little helpers --------
%a section frame: displays current section name plus some subcaption as separate slide
\newcommand{\sectionframe}[1]
{{
\setbeamercolor*{frametitle}{fg=normal text.fg, bg=normal text.bg} % set bg to green (or so) when adjusting..
\begin{frame}[c]
    \begin{center}
    \usebeamerfont*{frametitle}
    \usebeamercolor*[fg]{frametitle}
    \center \LARGE \arabic{section}. \insertsection \\[1em]
    \usebeamerfont*{subtitle}
    \usebeamercolor*[fg]{subtitle}
    \raggedright \normalsize
    #1
    \end{center}
\end{frame}
}}

\newcommand{\subsectionframe}[1]
{{
\setbeamercolor*{frametitle}{fg=normal text.fg, bg=normal text.bg} % set bg to green (or so) when adjusting..
\begin{frame}[c]
    %\begin{center}
    \usebeamerfont*{frametitle}
    \usebeamercolor*[fg]{frametitle}
    \Large \arabic{section}.\arabic{subsection} \insertsubsection \\[1em]
    \usebeamerfont*{subtitle}
    \usebeamercolor*[fg]{subtitle}
    \raggedright \normalsize
    \small #1
    %\end{center}
\end{frame}
}}


% ---------------------------------------------------------
% boxes
% ---------------------------------------------------------

\newsavebox{\selvestebox}
\newenvironment{colbox}[1]
  {\newcommand\colboxcolor{#1}\begin{lrbox}{\selvestebox}\begin{minipage}[t]{\dimexpr\columnwidth-2\fboxsep\relax}{}
  }
  {\end{minipage}
  \end{lrbox}%
   %\begin{center}
   \colorbox{\colboxcolor}{\usebox{\selvestebox}}
   %\end{center}
   }
\newenvironment{colboxw}[2]
  {\newcommand\colboxcolor{#1}\begin{lrbox}{\selvestebox}\begin{minipage}[t]{#2}{}
  }
  {\end{minipage}
  \end{lrbox}%
   %\begin{center}
   \colorbox{\colboxcolor}{\usebox{\selvestebox}}
   %\end{center}
   }

\definecolor{apricot}{RGB}{251, 206, 177}
\newcommand{\beispiel}[1]{\usebeamercolor*{block body}\adjustbox{margin=0.5ex,bgcolor=bg}{\small\color{fg}#1}}
%
\renewcommand{\beispiel}[1]{
\begin{minipage}{\linewidth}
\begin{block}{\vspace{-3cm}}%no title bar
#1
\end{block}
\end{minipage}}

\newenvironment{beispiele}
{\begin{minipage}{\linewidth} \begin{block}{\vspace{-3cm}}%no title bar
}
{\end{block}\end{minipage}
}

\newenvironment{variableblock}[3]{%
  \setbeamercolor{block body}{#2}
  \setbeamercolor{block title}{#3}
  \begin{block}{#1}}{\end{block}}
%\begin{variableblock}{Title}{bg=blue,fg=white}{bg=green,fg=red}
%  Stuff
%\end{variableblock}

\renewenvironment{example}{\begin{block}{Beispiel}}{\end{block}}
\newenvironment{observation}{\begin{block}{Beobachtung}}{\end{block}}

\newcommand{\getnw}{\tikz[na] \node (nw)[coordinate] at ($(current page.north west)+(0.8cm, -1.5cm)$) {};}
%\newcommand{\getorigin}{\tikz[na] \node[coordinate] (nw) {};}

% placing pictures
\newcommand{\myframeBR}[2]
{
\getnw
\begin{tikzpicture}[overlay]
\draw [above left] (nw) + (\textwidth, -\frameheight) node[inner sep=0mm]  {\begin{minipage}[b]{#1} #2 \end{minipage}};
\end{tikzpicture}
}
\newcommand{\mypictureBR}[3]
{
\myframeBR{#1}{\includegraphics[width=\textwidth]{#2} \\ \tiny #3}
}
\newcommand{\myframeTR}[2]
{
\getnw
\begin{tikzpicture}[overlay]
\draw [below left] (nw) + (\textwidth, 0) node[inner sep=0mm]  {\begin{minipage}[b]{#1} #2 \end{minipage}};
\end{tikzpicture}
}
\newcommand{\mypictureTR}[3]
{
\myframeTR{#1}{\includegraphics[width=\textwidth]{#2} \\ \tiny #3}
}
\newcommand{\myframeTL}[2]
{
\getnw
\begin{tikzpicture}[overlay]
\draw [below right] (nw)  node[inner sep=0mm]  {\begin{minipage}[b]{#1} #2 \end{minipage}};
\end{tikzpicture}
}
\newcommand{\mypictureTL}[3]
{
\myframeTL{#1}{\includegraphics[width=\textwidth]{#2} \\ \tiny #3}
}

\newcommand{\myframeBL}[2]
{
\getnw
\begin{tikzpicture}[overlay]
\draw [above right] (nw) + (0, -\frameheight) node[inner sep=0mm]  {\begin{minipage}[b]{#1} #2 \end{minipage}};
\end{tikzpicture}
}
\newcommand{\mypictureBL}[3]
{
\myframeBL{#1}{\includegraphics[width=\textwidth]{#2} \\ \tiny #3}
}

\tikzstyle{every picture}+=[remember picture]
\tikzstyle{na} = [baseline=-.5ex]
\newcommand{\frameheight}{\textheight}


\newcommand{\quellen}[1]{
\only<handout|trans>{
\begin{tikzpicture}[overlay,remember picture]
\draw [above left] (current page.south east) node[align=left]{\tiny\color{gray}\rotatebox{90}{\parbox[l]{20cm}{#1}}};
\end{tikzpicture}
}}
\newcommand{\lbl}[1]{\tikz[na] \node[coordinate] (#1) {};}
\newcommand{\lblb}[2]{\tikz[na]\node[inner sep=0pt](#1){\strut#2};}


% ---------------------------------------------------------
% listings
% ---------------------------------------------------------
\lstset{
language=C++,
basicstyle=\srcfont,
keywordstyle=\fcb,
commentstyle=\dimm,
stringstyle=\fce,
showstringspaces=false,
columns=flexible
%numbers=none, numberstyle=\tiny, stepnumber=1, numbersep=5pt, captionpos=b,
%columns=flexible % flexible, fixed, fullflexible
%framerule=1mm,frame=shadowbox, rulesepcolor=\color{blue}, % frame = shadowbox
%xleftmargin=2mm,xrightmargin=2mm
}
% list styles
% style: include @xxx@ and @@xxxx@@ for highlighting
\lstdefinestyle{highlight}{
moredelim=**[is][\hl]{@}{@},
moredelim=**[is][\hlx]{@@}{@@},
}
% style: fixed width columns: when character-wise alignment is absolutely crucial
\lstdefinestyle{fixed}{
basicstyle=\srcfont\small,
columns=fixed,
style=highlight
}
% style: normal
\lstdefinestyle{normal}{
basicstyle=\srcfont\small,
columns=flexible,
style=highlight
}
% style: normal with normalsize
\lstdefinestyle{normaln}{
basicstyle=\srcfont\normalsize,
columns=flexible,
style=highlight
}
% style: narrow: when more space is required horizontally
\lstdefinestyle{narrow}{
basicstyle=\srcfontn\small,
columns=flexible,
style=highlight
}
%style: narrow with normalsize
\lstdefinestyle{narrown}{
basicstyle=\srcfontn\normalsize,
columns=flexible,
style=highlight
}
\alt<handout>
{\lstdefinestyle{smallhandout}{basicstyle=\srcfont\footnotesize}}
{\lstdefinestyle{smallhandout}{}}
% overlay with highlighting
\lstdefinestyle{overlay}{
moredelim=**[is][\only<1|handout:0|trans:0>{\hl}]{@1@}{@1@},
moredelim=**[is][\only<2|handout:0|trans:0>{\hl}]{@2@}{@2@},
moredelim=**[is][\only<3|handout:0|trans:0>{\hl}]{@3@}{@3@},
moredelim=**[is][\only<4|handout:0|trans:0>{\hl}]{@4@}{@4@},
moredelim=**[is][\only<5|handout:0|trans:0>{\hl}]{@5@}{@5@},
moredelim=**[is][\only<6|handout:0|trans:0>{\hl}]{@6@}{@6@},
moredelim=**[is][\only<7|handout:0|trans:0>{\hl}]{@7@}{@7@},
moredelim=**[is][\only<8|handout:0|trans:0>{\hl}]{@8@}{@8@},
moredelim=**[is][\only<9|handout:0|trans:0>{\hl}]{@9@}{@9@},
moredelim=**[is][\only<1|handout:0|trans:0>{\hlx}]{@@1@}{@@1@},
moredelim=**[is][\only<2|handout:0|trans:0>{\hlx}]{@@2@}{@@2@},
moredelim=**[is][\only<3|handout:0|trans:0>{\hlx}]{@@3@}{@@3@},
moredelim=**[is][\only<4|handout:0|trans:0>{\hlx}]{@@4@}{@@4@},
moredelim=**[is][\only<5|handout:0|trans:0>{\hlx}]{@@5@}{@@5@},
moredelim=**[is][\only<6|handout:0|trans:0>{\hlx}]{@@6@}{@@6@},
moredelim=**[is][\only<7|handout:0|trans:0>{\hlx}]{@@7@}{@@7@},
moredelim=**[is][\only<8|handout:0|trans:0>{\hlx}]{@@8@}{@@8@},
moredelim=**[is][\only<9|handout:0|trans:0>{\hlx}]{@@9@}{@@9@}
}
% style: dim after step 1
\lstdefinestyle{dim}{
basicstyle=\srcfont\small\only<1>{\color{textfg}}\only<2-|handout:0|trans:0>{\color{textdimmed}},
commentstyle=\only<1>{\color{gray}}\only<2-|handout:0|trans:0>{\color{textdimmed}}
}
\lstdefinestyle{dimn}{
basicstyle=\srcfont\normalsize\only<1>{\color{textfg}}\only<2-|handout:0|trans:0>{\color{textdimmed}},
commentstyle=\only<1>{\color{gray}}\only<2-|handout:0|trans:0>{\color{textdimmed}}
}
\lstdefinestyle{dimmed}{
style=normal,
basicstyle=\srcfont\small\color{textdimmed}
}
\lstdefinestyle{cpp}{moredelim=**[is][\color{texthighlighted}]{@}{@},
moredelim=**[is][\color{texthighlightedx}]{@@}{@@},
basicstyle=\srcfont\normalsize}


% ---------------------------------------------------------
% algorithms
% ---------------------------------------------------------

\makeatletter
\renewcommand{\SetKwInOut}[2]{%
  \sbox\algocf@inoutbox{\KwSty{#2}\algocf@typo:}%
  \expandafter\ifx\csname InOutSizeDefined\endcsname\relax% if first time used
    \newcommand\InOutSizeDefined{}\setlength{\inoutsize}{\wd\algocf@inoutbox}%
    \sbox\algocf@inoutbox{\parbox[t]{\inoutsize}{\KwSty{#2}\algocf@typo:\hfill}~}\setlength{\inoutindent}{\wd\algocf@inoutbox}%
  \else% else keep the larger dimension
    \ifdim\wd\algocf@inoutbox>\inoutsize%
    \setlength{\inoutsize}{\wd\algocf@inoutbox}%
    \sbox\algocf@inoutbox{\parbox[t]{\inoutsize}{\KwSty{#2}\algocf@typo:\hfill}~}\setlength{\inoutindent}{\wd\algocf@inoutbox}%
    \fi%
  \fi% the dimension of the box is now defined.
  \algocf@newcommand{#1}[1]{%
    \ifthenelse{\boolean{algocf@inoutnumbered}}{\relax}{\everypar={\relax}}%
%     {\let\\\algocf@newinout\hangindent=\wd\algocf@inoutbox\hangafter=1\parbox[t]{\inoutsize}{\KwSty{#2}\algocf@typo\hfill:}~##1\par}%
    {\let\\\algocf@newinout\hangindent=\inoutindent\hangafter=1\parbox[t]{\inoutsize}{\KwSty{#2}\algocf@typo:\hfill}~##1\par}%
    \algocf@linesnumbered% reset the numbering of the lines
  }}%
\makeatother

\SetKwInOut{KwInput}{Input}
\SetKwInOut{KwOutput}{Output}
\SetAlFnt{\algfont\small}
\SetKwSty{foffont}
\SetArgSty{algfont}
\newcommand{\commentstyle}{\small\algcfont\color{gray}}
\SetCommentSty{commentstyle}
\newcommand{\vlinecolor}{gray}
\SetKw{KwDownTo}{downto}
\SetKw{KwAnd}{and}
\SetKw{KwOr}{and}
\DontPrintSemicolon
\newcommand{\variable}[2]{\newcommand{#1}{\mathit{#2}}}
\newcommand{\alignedcomment}[2][0.5]{\tcp*[f]{\makebox[#1\textwidth]{#2\hfill}}}


\SetNlSty{commentstyle}{}{}
% vertical rules in grey color
\makeatletter
% it is very strange: but this vlinecolor does not work. Tried a lot of different attempts. Leave it for the time being...
\renewcommand{\algocf@Hlne}{{\color{\vlinecolor}\hrule height 0.4pt depth 0pt width .5em}}%

%%%%%%%%% block with a vertical line end by a little horizontal line
\renewcommand{\algocf@Vline}[1]{%     no vskip in between boxes but a strut to separate them,
  \strut\par\nointerlineskip% then interblock space stay the same whatever is inside it
  \algocf@push{\skiprule}%        move to the right before the vertical rule
  \hbox{{\color{\vlinecolor}\vrule}%
    \vtop{\algocf@push{\skiptext}%move the right after the rule
      \vtop{\algocf@addskiptotal #1}{\algocf@Hlne}}}\vskip\skiphlne% inside the block
  \algocf@pop{\skiprule}%\algocf@subskiptotal% restore indentation
  \nointerlineskip}% no vskip after
%
%%%%%%%%% block with a vertical line
\renewcommand{\algocf@Vsline}[1]{%    no vskip in between boxes but a strut to separate them,
  \strut\par\nointerlineskip% then interblock space stay the same whatever is inside it
  \algocf@bblockcode%
  \algocf@push{\skiprule}%        move to the right before the vertical rule
  \hbox{{\color{\vlinecolor} \vrule}%               the vertical rule
    \vtop{\algocf@push{\skiptext}%move the right after the rule
      \vtop{\algocf@addskiptotal #1}}}% inside the block
  \algocf@pop{\skiprule}% restore indentation
  \algocf@eblockcode%
}
%
%
%
\makeatother
%
%
% ---------------------------------------
%  tables
% ---------------------------------------

\colorlet{tableheadcolor}{bcb!75} % Table header colour = 25% gray
\newcommand{\headcol}{\rowcolor{tableheadcolor}} %
\colorlet{tablerowcolor}{bcb!50} % Table row separator colour = 10% gray
\newcommand{\rowcol}{\rowcolor{tablerowcolor}} %
\newcommand{\topline}{\arrayrulecolor{fcb}\specialrule{0.1em}{\abovetopsep}{0pt}%
            \arrayrulecolor{tableheadcolor}\specialrule{\belowrulesep}{0pt}{0pt}%
            \arrayrulecolor{fcb}}
\newcommand{\midline}{\arrayrulecolor{tableheadcolor}\specialrule{\aboverulesep}{0pt}{0pt}%
            \arrayrulecolor{fcb}\specialrule{\lightrulewidth}{0pt}{0pt}%
            \arrayrulecolor{fcb}}
\newcommand{\bottomline}{\arrayrulecolor{bg}\specialrule{\aboverulesep}{0pt}{0pt}%
            \arrayrulecolor{fcb}\specialrule{\heavyrulewidth}{0pt}{\belowbottomsep}}%

% ---------------------------------------------------------
% miscellaneous
% ---------------------------------------------------------

% a vertical separator
\newcommand{\vsep}{\vspace{1em}}

% really often used mathematical symbols
\newcommand{\R}{\mathbbm{R}}
\newcommand{\N}{\mathbbm{N}}
\newcommand{\Z}{\mathbbm{Z}}
\newcommand{\C}{\mathbbm{C}}
\newcommand{\Q}{\mathbbm{Q}}
\newcommand{\I}{\mathbbm{1}}
\newcommand{\prob}{\mathbbm{P}}
\newcommand{\vP}{\mathbbm{P}}
\newcommand{\expect}{\mathbbm{E}}
\newcommand{\vE}{\mathbbm{E}}
\newcommand{\calK}{\ensuremath{\mathcal{K}}}
\newcommand{\calH}{\ensuremath{\mathcal{H}}}
% really often used language names
\newcommand{\cpp}{\textrm{C{+}{+}}}
\newcommand{\plainc}{\textrm{C}}

\newcommand{\AND}{\mathop{\mathrm{AND}}}
\newcommand{\OR}{\mathrm{OR}}
\newcommand{\NOT}{\mathrm{NOT}}
\newcommand{\XOR}{\mathrm{XOR}}
\newcommand{\ratpack}{RAT PACK$^\circledR$}

% a smiley
\newcommand{\happy}{\tikz[baseline=-0.75ex,black,very thick,fill=bce,draw=fce,scale=1.2]{
    \draw[fill,draw] circle (2mm);
\node[fill=fce,circle,inner sep=0.5pt] (left eye) at (135:0.8mm) {};
\node[fill=fce,circle,inner sep=0.5pt] (right eye) at (45:0.8mm) {};
\draw (-145:0.9mm) arc (-140:-40:1mm);
    }}
\newcommand{\sad}{\tikz[baseline=-0.75ex,black,very thick,fill=bca,draw=fca,scale=1.2]{
    \draw[fill,draw] circle (2mm);
\node[fill=fca,circle,inner sep=0.5pt] (left eye) at (135:0.8mm) {};
\node[fill=fca,circle,inner sep=0.5pt] (right eye) at (45:0.8mm) {};
\draw (-35:1.2mm) arc (40:140:1.2mm);
    }}
\newcommand{\boring}{\tikz[baseline=-0.75ex,black,very thick,fill=bcc,draw=fcc,scale=1.2]{
    \draw[fill,draw] circle (2mm);
\node[fill=fcc,circle,inner sep=1pt] (left eye) at (135:0.8mm) {};
\node[fill=fcc,circle,inner sep=1pt] (right eye) at (45:0.8mm) {};
\node[fill=bcc,inner sep=1.2pt,above] at (left eye) {};
\node[fill=bcc,inner sep=1.2pt,above] at (right eye) {};
\node[draw=fcc,circle,inner sep=0.5mm] (mouth) at (-70:1.1mm) {};
    }}



\newcommand{\todo}[1]{\tikz[overlay]{\getnw;\node[draw,rectangle,fill=red,right] at (nw){\color{white}! #1 !};}}
\newcommand{\bigO}{\ensuremath{\mathcal{O}}}
\renewenvironment{proof}{Beweis: }{\hfill $\blacksquare$}
\newcommand{\floor}[1]{\ensuremath{\left\lfloor#1\right\rfloor}}
\newcommand{\ceil}[1]{\ensuremath{\left\lceil#1\right\rceil}}
\newenvironment{question}[1]
{\begin{variableblock}{
\tikz[]{\node[circle,draw, inner sep= 1pt,ultra thick] at (0,0.2) {\bf ?};} #1}{bg=textbg,fg=textfg}{bg=flatboxcolor,fg=textfg}
}
{\end{variableblock}}
\newcommand{\answer}{\tikz[]{\node[circle,draw, inner sep= 1pt,ultra thick] at (0,0.2) {\bf !};}}
\newcommand{\questionM}{\tikz[]{\node[circle,draw, inner sep= 1pt,ultra thick] at (0,0.2) {\bf ?};}}
%  Stuff
%\end{variableblock}

\newcommand{\Widmayer}[1]{Ottman/Widmayer, Kap.\ #1}
\newcommand{\Cormen}[1]{Cormen et al, Kap.\ #1}
\newcommand{\Stroustrup}[1]{Stroustrup, Kap.\ #1}
\newcommand{\Herlihy}[1]{Herlihy/Shavit, Kap.\ #1}

\graphicspath{{./img/}{./}}
\newcommand{\mypicture}[2][1.0]{
    \includegraphics[width=#1\textwidth, height=#1\textheight, keepaspectratio]{#2}
}

\ifdefined\englishSlides
\newcommand\trans[2]{#2}
\else
\newcommand\trans[2]{#1}
\fi
%\newcommand{\trans}[2]{\ifdefined\englishSlides #2\else #1 \fi}

